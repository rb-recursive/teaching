\documentclass[authoryear,11pt]{elsarticle}

%This eliminates the `Preprint submitted to...' footer on the first page
\makeatletter
\def\ps@pprintTitle{%
 \let\@oddhead\@empty
 \let\@evenhead\@empty
 \def\@oddfoot{}%
 \let\@evenfoot\@oddfoot}
\makeatother

\usepackage{amssymb}
\usepackage{amsthm}
\usepackage{caption}
\usepackage{amsmath}
\usepackage{morefloats}
\usepackage{bbm}        %To allow \mathbb{1}

\usepackage{rotating}   %To turn tables sidewaystable
\usepackage{graphicx}
\usepackage{setspace}
\usepackage{hyperref}

%\onehalfspacing

%\setlength{\parindent}{0pt}

\usepackage[top=3.5cm,bottom=3.75cm,left=2.45cm,right=2.45cm]{geometry}% by courtesy of Mico

\begin{document}
\begin{frontmatter}
\title{MFE Economics\\Lecture 4: Intuition}
\end{frontmatter}

%%%%%%%%%%%%%%%%%%%%%%%%%%%%%%%%%%%%%%%%%%%%%%%%%%%%%%%%%%%%%%%%%%%%%%%%%%%%%%%%%%%%%%%%%%%%%%%%%%%%%%%%%%%%%%%%%%%%%%%%%%%%%%%%%%%%%%%%%%%%%%%%%%%%%%%%
%%%%%%%%%%%%%%%%%%%%%%%%%%%%%%%%%%%%%%%%%%%%%%%%%%%%%%%%%%%%%%%%%%%%%%%%%%%%%%%%%%%%%%%%%%%%%%%%%%%%%%%%%%%%%%%%%%%%%%%%%%%%%%%%%%%%%%%%%%%%%%%%%%%%%%%%

\section{Basic New Keynesian Model - Households, Goods and Firms}
In the models we have examined previously, we had perfectly competitive firms, producing an identical consumption good. In the NK model we introduce multiple goods (indexed by $i$), each produced by a different, monopolistically competitive firm. Each firm faces a `downward sloping' (rather than `horizontal' as in perfect competition) demand curve meaning that its demand varies smoothly with its relative price (they don't lose all their sales if they price marginally above a market price as in the competitive case).

Each firm chooses its price, in contrast to the price taking assumption of perfectly competitive analysis. Immediately this should hint at the likelihood of inefficiency in the ultimate equilibrium - which opens the door to the possibility there may be a role for welfare-improving policy intervention. Additionally, since a firm does not lose all its customers if its price is slightly out of line and because we explicitly allow firms to be price setters, imperfect competition allows for a meaningful role for sticky prices - which have been regarded over many decades of monetary economics as vital in generating real effects of policy.

The various goods enter the household's utility function, symmetrically. The household exhibits equal cross-price elasticity of substitution between all goods (controlled by $\varepsilon$) and \textit{ideally} would like to consume the same amount of each good (the Dixit-Stiglitz aggregator is concave so they dislike dispersion). However, if relative prices make consuming a particular good in larger or smaller relative quantities, the extent to which the consumer substitutes toward/away from it will depend on $\varepsilon$. Thus, when we derive the individual good demand curve (which the firms then take as their demand curve) we see that optimal demands, for a given total consumption bundle, are negatively related to relative price, with the elasticity given by $-\varepsilon$. Again, it is convenient for each to be modeled as accounting for an infinitesimally small part of the bundle and its price have no individual effect on the aggregate price level - so we use again $i\in[0,1]$ for convenience - we say there is `measure 1 of gods $i$'.

Happily, if we \textit{define} the aggregate price level $P_{t}$ appropriately - building it up as a particular sort of average of the individual prices $P_{t}^{i}$) we can write total expenditure on the goods (by the representative household) as $P_{t} C_{t}$ where $C_{t}$ is the bundle of the individual goods consumed by the household and which enters its felicity function as it did in our earlier models (where there was only one good to begin with). This means that a lot of our math (the BC, the FOCs) look much as before.

Firms are very much like before, in terms of their production functions - though again we note there are (uncountably infinitely) many `small' firms. They are price takers in the labor market and \textit{price makers} in their own product market. They are monopolistic in their own market but - like a producer of a particular beer - there are other similar goods that are quite substitutable, so they lose some business as they raise their prices (this is implicitly captured in the form of the demand curve and through the value of $\varepsilon$ - note that $\varepsilon \rightarrow \infty$ is like a price taking / competitive limit.\footnote{We abstract from worrying about long run equilibrium in a monopolistically competitive setup, which involves entry until a zero profits steady state is obtained - with decreasing returns to scale this becomes a bit weird and NK models rarely discuss it. I think what happens in that case is we imagine a limiting case where each firm is becoming arbitrarily small - but I'm not sure. You do not need to know this. The treatment of profits and industrial structure are not strong points of the NK model.}

In one - maybe two - of the homeworks we discuss a standard monopolistically competitive (MonC) case where optimal pricing implies a markup over nominal marginal cost
\[
P_{t}(i)^{\ast} = \mathcal{M} MC_{t} \equiv \frac{\varepsilon}{\varepsilon-1}
\]
where marginal cost is the nominal wage divided by the marginal product of labor. Why? You need 1/MPL extra labor to produce a marginal unit of output (obviously - right?) and you need to pay that marginal extra labor at the wage rate (which you take as given). Note - as discussed at length in the answers to one of the early problem sets, we do not recover $P_{t}=MC_{t}$ except in the limit as $\varepsilon\rightarrow \infty$, implying there is an inefficiency in which output is constrained. This happens because in choosing their price (which is equivalent to choosing output, via the demand curve, and employment, via the production function) the firms are taking account of the downward sloping demand curve - the fact that they need to charge a lower price to sell more but that the lower price is not only on the marginal additional unit to be sold, but on all the `inframarginal' units that they could have been selling at a higher price.

\section{Price stickiness}
In the data (see the references in Gali chapter 1) there is much evidence of various forms of price stickiness. Prices prevail for several periods, rather than jumping around with every slight change in information and context that the firm faces. The stickiness of prices opens the door to monetary non-neutrality in the short run (remember the $MV=PY$ and Fisher equation discussion). There are various ways of modeling the stickiness of prices - e.g. by assuming there is some cost to changing (`menu costs') that mean you only change your price once it has drifted `too far' from optimal. 

In our simple case, we adopt a common reduced form called `Calvo pricing' where we assume that every period each firm has a probability $1-\theta$ (with $\theta \in [0,1)$) of a firm being able to reset/reoptimize its price. Being able to reset a price in $t$ is identically and independently distributed across firms and across time. How long your price has been prevailing has no impact on the probability of being allowed to change the price (very unrealstic, but hey). Since there are uncountably infinitely many firms this means that in each period a fraction $\theta$ of firms are unable to change their price. The expected (mean) duration of a price is $(1-\theta)^{-1}$ which you can take to the data to pin down $\theta$. That is, if the average duration is $8$ months and your model is monthly, you would set $(1-\theta)^{-1} = 8$ and rearrange. You should be comfortably manipulating the expressions involving $\theta$ I put in the problem sets and understand how these assumptions lead to there being price stickiness \textit{and} price dispersion (there are a bunch of firms who haven't reset their prices for various lengths of periods).

The main thing, though, to bear in mind is that the existence of a distribution of firms' prices, the knowledge that not all firms will be able to reset in the next or future periods and the awareness that you, firm $i$, may not be able to reset, changes the firms' pricing problem massively. They have to become forward looking, rather than simply solving a one shot problem (always implies $\mathcal{M}$ markup in all periods). Essentially you want to keep `close to' the one shot $\mathcal{M}$ markup in some sense, but the price that achieves that in the period you reset might - given your beliefs about the probabilty of resetting and where the aggregate price level (which will then determine your relative price, demand, hiring and markup in the future - given your fixed price) is going - end up being horribly costly in other periods and contingencies in which is continues to prevail. So you have to trade those costs and benefits off and this will typically (unless the economy is in steady state with zero price dispersion) induce you to violate the one shot optimality condition.

As the expression on p. 20 shows, the firms (on behalf of the household) are in some sense choosing what asset to invest in - by choosing $P_{t}(i)^{\ast}$ you are determining a sequence of (random) profits (which are valued today using the household's discount factor - since households own the firms) given the probabilities of your price lasting different lengths of time. The slides are then pretty clear (also you should read Gali) on the pricing problem and optimality condition (differenitate wrt $P_{t}(i)^{\ast}$, set equal to zero and rearrange and manipulate like crazy\ldots). Ultimately the condition (4) on page 26 shows that firms are still marking up, but not on current marginal cost alone (unless $\theta = 0$ - what's the interpretation of that?) but on a particular average of marginal costs over the future - weighted according to their likelihood and time preference.\footnote{Note that in the linearized world - we have approximated around the zero inflation steady state - the risk parts of the SDF - stuff involving marginal utility - have disappeared. This is a (undesirable) property of $1^{st}$ order approximations. The true model would have risk present - and there would be risk premia implicitly involved but that is beyond the scope of this course, though you probably are learning a lot about this in other classes where its not only expected payoffs that matter, but their covariance with your pricing kernel.}

Note also that, although the firms are not able to guarantee (due to the probability of being unable to reset) that they will maintain the flex price markup condition ($P_{t} = \mathcal{M}\Psi_{t}$) you can see in expression (3) that this is informing their decision, in a sense, as they are setting prices to keep an `average' of deviations from this condition, close to zero. And, again, note that if $\theta=0$ (the flex price case) we covert the flex price markup condition.

You should be comfortable (see the Warwick final exam) with the fairly simple manipulations that go from individual price resetters' price choices (the same for all price setters in a given period) and the prices of the non-resetters, to the behavior of aggregate inflation (slides around p. 27).

\section{Equilibrium - non-policy block}
Here (slide 30) we begin a long sequence of derivations that are intended to connect a measure of real activity (for us it will be an output gap) with price behavior - ending with the New Keynesian Phillips curve, which, together with the DIS curve are two of the three `main equations' in simple NK models (the other one being a policy rule, discussed in the next section).\footnote{These are fairly tedious slides - though you should be comfortable understanding each step in case I ever give you a small number of consecutive steps and ask you to go  between them. You do not need to worry about the approximation mentioned on p. 30).}

You should bear in mind that connecting prices (inflation) to output entails thinking about how to connect marginal costs to output (wages, marginal products/decreasing returns to scale at the firm level and market clearing consumption=output and labor demand=supply at the GE level) and how to connect marginal costs to prices (the markup - are firms happy with their prices, what re-setters when allowed to reset). It can be easy to lose sight of the wood, for the trees, but this is about pricing behavior - which is obviously driven by costs, pricing relative to costs and, in this setup, expectations of the future.

On p.32 we see that in our case, non-zero inflation is, \textit{in equilibrium}, associated with a non-zero discounted sum of expected deviations of markups from desired. So, prices (inflation captures the change of prices, right?) will be changing because the resetters realize that things are `out of line' so they start adjusting. They don't go all the way there in one go (they take account that they might be constrained in future periods, say) but the gap starts to be eliminated: if the gaps are negative (markups typically projected to be too low in the future) then it means that (given beliefs about future costs - which are co-determined) prices are expected to be `too low' - so they start to rectify that by raising prices (note that $\lambda>0$). If the price setters are changing their prices, then inflation is non-zero. This is where you should be going back to homeworks where you've been asked about what parameters affect $\lambda$ and think about whether it all makes sense.

The handful slides from slide 32 onward (which I emphasized in a homework) look fairly dry but, as noted in the homework, there's a lot that is educational here. We see how we are starting to eliminate variables (basically the process to `solve' the model) by substituting in equations based on equilibrium conditions (mkt. clearing and optimality, plus use of the production function, etc.). Additionally, we see that even with CRS ($\alpha =0$) where at the (wage-taking) firm level there is still a positive relationship \textit{\textbf{in equilibrium}} between output and real marginal cost. This arises because the labor market must clear and to induce more labor to be supplied, \textit{given we respect household optimal labor supply conditions} the market clearing wage must be higher.\footnote{Again, we are silent on `how' this wage increase occurs as everyone is a price taker in the labor market - we simply note that for the economy to be in equilibrium we \textit{need} the wage to be higher.}

\subsection{Natural rate of output and the NKPC}
The `natural' value of a real variable is the value that obtains in the \textit{absence of price stickiness}. That is, if we set $\theta=0$ so there is no probability of a firm being unable to change its price next period. In our context, that means that firms are always able to set prices relative to marginal cost to achieve the desired markup, $\mathcal{M}$. Recall, this price then determines output (via the demand curve they face - recall the first part of the lecture) so firms are always operating at their desired scale. The `natural output' level for then reflects this. Note that the natural output level will not be constant since it still depends on the technology shock, $a_{t}$. Fluctuations in technology will shift the natural rate over time.\footnote{In a model with capital accumulation - i.e. a more realistic model - the preference shock would have an effect too.}

Why bother with this (and other `natural' variables)? Partly it's because once you have defined the associated flex price economy (a `shadow economy') it makes clear the effects of sticky prices and monetary policy. Additionally, it turns out that it helps make the equations tidier (useful to work in terms of $\tilde{y}_{t}\equiv y_{t}-y_{t}^{n}$), which is primarily a clarity issue. However, there is also some intuition: if actual $y_{t}$ is different from $y_{t}^{n}$ you know that the economy is somehow `out of kilter' (though it's still in equilibrium - think about this distinction), there is price dispersion, markups aren't at the `desired' level and price setters who can adjust their prices will do so, implying nonzero inflation, as discussed above.

If the only distortion in the economy is the competitive distortion, and that is solved with a subsidy that lowers MC until firms choose to produce the socially optimal output level, then the policymaker will want output to equal the natural rate. In this case there is a coincidence between wanting output to be at natural and inflation to be zero.\footnote{There is a minor issue about initial conditions for price dispersion - discussed in Gali - but ignore for now. Also, in the real world, the flex price of `natural' level of output may not be ideal even if it was possible to solve the steady state competitive distortion with a hiring subsidy like that discussed in Gali and your homework. Hence, policymakers don't typically slavishly pursue zero inflation period to period to keep $y_{t}=y^{n}_{t}$ and they trade off inflation and output volatility.}

Another point to note is that the steady state values of $y_{t}$ and $y_{t}^{n}$ are the same in a zero inflation world (whether we have a constant subsidy or not) as the `steady state' or `long run average' value of $y_{t}$ entails imagining what would happen once the effect of any transitory price dispersion has died out, once firms have all managed to reset `enough times'.\footnote{There's a little subtlety here as we haven't yet got to monetary policy or a zero inflation target that, whenever we take approximations (e.g. slide 27), is the point around which we approximate. To go into this further will be well beyond the scope of this course though I touch on it again when we discuss trade-offs between inflation and output.}

Since $\theta=0$ is simply a special case of our sticky price model we can use the same structural equations. In particular, around slide 37 we imagine the flex price case and, recalling that it means $\mu_{t} = \mu \equiv \log \mathcal{M}$, we can \textit{define} $y_{t}^{n}$ and express its equilibrium value as a function of the state (in this case it only depends on the $a_{t}$ part of the state).

Using these derivations for the (imaginary) natural case and the (actual) price stickiness case we can rearrange and subtract various equations to obtain the \textit{New Keynesian Phillips Curve} relating the output gap ($\tilde{y}_{t}\equiv y_{t}-y_{t}^{n}$) to current and future expected inflation.
\begin{equation}
\pi_{t} = \beta E_{t}[ \pi_{t+1} ] + \kappa \tilde{y}_{t}
\end{equation}

Note this is a structural equation that must hold in equilibrium - it is not a causal statement as $\pi_{t}$, $\tilde{y}_{t}$ and $E_{t}[\pi_{t+1}]$ are all endogenous variables. It will be used together with other equilibrium conditions and assumptions about the random innovations to the economy, to eventually solve for all endogenous variables in terms of the state.

As you will find in doing your `research' in homework 5, this equation continues a long tradition in macroeconomics of trying to explain (justify?) the - supposed - tendency for there to be a relationships between price inflation and some measure of real activity. Unlike early generations which only used reduced form correlations and/or dubious modeling of pricing behavior and expectations formulation, we have derived this equation `the hard way' based on microfoundations and under the demanding conditions (though I haven't talked about this much) of `Rational Expectations'.

We see that there is \textit{some form of association} between strong activity (positive output gap) and inflation, which people traditionally believe exists, reflected in positive $\kappa$ but note that whether or not in the data there is a positive correlation will depend on how $E_{t}[\pi_{t+1}]$ moves. Given expected future inflation, there is a positive relationship (though many people argue that $\kappa$ is currently close to zero - a `flat' Phillips curve) between $\tilde{y}_{t}$ and $\pi_{t}$ but that `given' is important.

Note also that the forward looking nature of the agents in our model is also reflected in the NKPC. If inflation is expected to be high, all else equal (not a full GE argument as we're holding $\tilde{y}_{t}$ fixed), then inflation will be high. This is a big deal as it shows that expectations are pivotal in determining \textit{current} inflation (and everything else actually). Iterating on the equation we see that current inflation can be expressed in terms of an expected path into the future of output gaps. Also, note that if output gaps are currently and expected to be zero, we must have $\pi_{t}=E_{t}[\pi_{t+1}]=0$ and, setting aside shocks - or thinking in terms of the long run - that means $\pi_{t}=\pi_{t+1}=0$.\footnote{Recall the earlier footnote about log-linearizing around the zero inflation steady state. Had we imagined a non-zero inflation rate that the policymaker was pursuing (the policy rule would be adjusted also) then we would have obtained a similar expression but in terms of inflation gaps - from target - where inflation only appears in our expression. So you can really think of the inflation terms as gaps (in this case the gap $\pi_{t}-0$ is just $\pi_{t}$ anyway). This means that it is the gap from inflation that must be closed in association with zero current and future output gaps. So in the long run, any constant target inflation level is associated with zero output gap - there is no long run trade-off between inflation and real activity. The classical dichotomy holds in the LR in the NK model (as it should).}
	
\subsection{Natural rate of interest}
If we impose `good market clearing' ($c_{t} = y_{t}$) and substitute it in to the representative household's Euler equation we obtain a useful expression. Again, imagining a flex price world - and noting that by the Fisher relation $r_{t}=i_{t}-E_{t}[\pi_{t+1}]$ - we can \textit{define} the natural real rate of interest as thay which prevails in the flex price economy. To derive it as a function of the state in equilibrium, we need to subtract the equation under flex price  from the `actual economy' (with sticky prices) equivalent and then use the (already derived) expression for the output gap in terms of the state, along with out knowledge of how to use expectations of AR(1) processes.\footnote{Make sure you know how to get equation (6).} We thus obtain the `Dynamic IS' (DIS) curve.\footnote{The name comes from the old - not microfounded IS curve - Keynesian tradition and `dynamic' I guess reflects that fact that it makes explicit the fact that it comes from intertermporal choice and also explicitly features expectations about the future.}

Iterating on the DIS (and using the law of iterated expectations) we can show the association between the output gap and expectations of current and future real rate gaps. Check that you understand where the $E_{t}[\tilde{y}_{t+k}]$ goes as $k \rightarrow \infty$.\footnote{Hint: I a stationary world, the conditional expectation of a variable infinitely in the future is equal to its unconditional expectation. Now recall what the unconditional expectation of the tech shock is \textit{and} the fact that steady state output and natural output are equal.}

At this point (slide 49 and subsequent) we can pull together all the equations defining (implicitly - we haven't yet solved for the endogenous variables in terms of the state - they're still defined in terms of eachother!) the equilibrium values of the real variables. In this economy, though, we need also to specify a rule for monetary policy since the short run neutrality of money does not hold, $i_{t}$ affects real variables in equilibrium and $i_{t}$ shows up in the dynamic IS.

\subsection{Introducing policy}
We will introduce policy by imagining a type of Taylor rule for $i_{t}$, where policy is posited to respond to inflation and (another) output gap, plus an AR(1) monetary policy shock, $v_{t}$. The policymaker affects the economy by setting $i_{t}$ and - due to price stickiness - thus affecting $r_{t}$ which, as the DIS shows, is connected to real activity, which, as the NKPC shows is connected to inflation behavior. It is standard to assume the inflation response coefficient $\phi_{\pi}>1$ (see later lectures for more on the Taylor principle).\footnote{Feel free to Google Taylor rule, Taylor principle and look at some of the readings.}

Note that we allow for the policymaker to be pursuing a rule that responds to output gaps \textit{relative to its steady state value} (which it shares with $y_{t}^{n}$). I can't remember if Gali give a justification for introducing more notation.\footnote{Not that it's difficult - just add and subtract $y_{t}^{n}$ and you get the expression on slide 53. Adding $\hat{y}_{t}$ is just like adding an equation and one more unknown to our system.} I imagine it reflects the assumption that long run output (given the assumed MC subsidy) is efficient or it's to reflect that fact that policymakers are unlikely to be able to track $y_{t}^{n}$ in the short run, and thus it is impractical to imagine policy responding to that $\tilde{y}_{t}$.

\end{document}

