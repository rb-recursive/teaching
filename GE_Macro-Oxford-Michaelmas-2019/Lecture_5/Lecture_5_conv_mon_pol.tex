%-------------------------------------------------------
\section{Monetary Policy in the New Keynesian Model}
%-------------------------------------------------------

\begin{frame}

\begin{center}
{\LARGE Monetary Policy in the NK Model}
\end{center}

\end{frame}

%-------------------------------------------------------

%-------------------------------------------------------

\begin{frame}{Monetary Policy in the New Keynesian Model}

\begin{itemize}
\item	We have shown that monetary policy can have real effects in the New Keynesian model
\item	We also note that the New Keynesian model features distortions that lead to inefficiencies
\item	Ability to affect an economy featuring inefficiencies $\Rightarrow$ possible role for policy interventions
\item	Central banks are heavy users (and developers of) New Keynesian models
\item	Structural models allow policymakers to understand and plan the effects of policy
\end{itemize}

\end{frame}

%-------------------------------------------------------

%-------------------------------------------------------
	
\begin{frame}{Steady state (in)efficiency in the NK model}

Recall from the previous lecture:
\begin{itemize}
\item	Steady state value of $y^{n}_{t}$ (and $y_{t}$) was lower than in Classical model
\item	Related to the pricing power of monopolistically competitive firms
\item	Markup: $\mu \equiv \log{(\mathcal{M})} \equiv \log{(\frac{\varepsilon}{\varepsilon-1})}$
\end{itemize}

\vspace{2mm}
Natural rate of output in NK model
\begin{eqnarray*}
y^{n}_{t} 	&=& 		\psi_{yn} + \psi_{yn,a} a_{t} \\
\psi_{yn}	&\equiv& 	-\frac{(1-\alpha)(\textcolor{red}{\mu}-\log{(1-\alpha)})}{\sigma(1-\alpha)+\varphi+\alpha}
\end{eqnarray*}

Output in Classical model
\begin{eqnarray*}
y^{c}_{t} 		&=& 		\psi_{yc} + \psi_{yc,a} a_{t} \\
\psi_{yc} 		&\equiv& 	\frac{(1-\alpha)\log{(1-\alpha)}}{\sigma(1-\alpha)+\varphi+\alpha} \\
\end{eqnarray*}

\end{frame}

%-------------------------------------------------------

%-------------------------------------------------------
	
\begin{frame}[label=efficiency]{Inefficiency in the NK model}

Monetary policy is not the `right' tool to correct industrial structure inefficiencies
\begin{itemize}
\item	Can't fix monopolistic competition
\item	Need to rely on supply side (e.g. tax/subsidy policies)
\item	In an earlier homework we showed that if producers are subsidized appropriately, we can recover the efficient level of output \hyperlink{efficiency_escape}{\beamergotobutton{More detail}}
\item	We will assume such a policy holds, so the \textbf{`steady state'} value of $y_{t}$ and both the steady state \textit{and current values} of $y^{n}_{t}$ are efficient
	\begin{itemize}
	\item	Flexible prices in the `natural rate' world $\Rightarrow$ if we fix competitive distortions, there's nothing left to cause inefficiency, in or out of steady state
	\end{itemize}
\end{itemize}

\vspace{2mm}
But that's not the end of the story\ldots

\end{frame}

%-------------------------------------------------------

%-------------------------------------------------------
	
\begin{frame}{Inefficiency in the NK Model}

Additional conditions that must be satisfied by an efficient allocation are\ldots

\vspace{2mm}
\begin{itemize}
\item	All goods (indexed by $i$) should be consumed (and thus produced) in the same quantities
\[
C_{t}(i) = C_{t}\text{  } \forall i \in [0,1]
\]

\item	All firms (each identified with a good, $i$) should employ the same amount of labor
\[
N_{t}(i) = N_{t}\text{  } \forall i \in [0,1]
\]

\end{itemize}

\end{frame}

%-------------------------------------------------------

%-------------------------------------------------------
	
\begin{frame}{An efficient benchmark}

Under the flex-price equilibrium or the zero inflation steady state of the sticky price model
\vspace{2mm}
\begin{itemize}
\item	All firms are producing the same amounts
	\begin{itemize}
	\item	Why? They face the same technology and prices
	\end{itemize}
\item	Each (identical) household consumes the same amount of each good
	\begin{itemize}
	\item	Why? All goods have the same price and enter symmetrically in the concave utility function
	\end{itemize}
\end{itemize}

\vspace{3mm}
But price stickiness prevents this from holding \textit{outside the steady state} from period to period

\end{frame}

%-------------------------------------------------------

%-------------------------------------------------------
	
\begin{frame}{An efficient benchmark}

Even if the \emph{steady state} of the NK model is efficient under the subsidy this does not mean it is efficient in any given period

\vspace{2mm}
\begin{itemize}
\item	Due to price stickiness, the average markup will vary over time
	\begin{itemize}
	\item	Average marginal cost will vary with average scale of production
	\item	Prices do not adjust fully to reflect this
	\item	Even if a constant subsidy is adjust for steady state markups, it can't correct for \textit{variations} in markups
	\item	Remember markups ($P \neq MC$) are associated with inefficiency
	\end{itemize}
	\vspace{2mm}
\item	\textbf{Due to price stickiness there will be dispersion in prices}
	\begin{itemize}
	\item	Leads to dispersion in consumption and employment across firms/goods
	\item	Violates optimality conditions for consumption and resource allocation
	\item	But this \textbf{\textit{is}} something monetary policy can rectify, in theory
	\end{itemize}
\end{itemize}

\end{frame}

%-------------------------------------------------------

%-------------------------------------------------------
	
\begin{frame}{Optimal Allocation}

Suppose we start from a steady state situation
\begin{itemize}
\item	All firms were setting the same price in the previous period
\item	Price was at desired markup over (subsidy adjusted) marginal cost
\item	All firms operate on the same scale
\item	Goods are consumed in the same quantity
\item	Output is at its natural level
\end{itemize}

\end{frame}

%-------------------------------------------------------

%-------------------------------------------------------
	
\begin{frame}{Optimal Allocation}

If shocks hit the economy, how should policy respond?
\begin{itemize}
\item	The aim is to preserve $y_{t} = y^{n}_{t}$ (or $\tilde{y}_{t}=0$) since $y_{t}^{n}$ is efficient
\item	Since this must be part of an equilibrium, the NKPC must hold
\item	Iterating the NKPC forwards $\Rightarrow$ if $ \tilde{y}_{t}=0$ $\forall t$ then $\pi_{t}=0$ $\forall t$
\end{itemize}
\begin{eqnarray*}
\pi_{t} 	&=& \beta E_{t}[\pi_{t+1}] + \kappa \tilde{y}_{t}	\\
		&=& \beta E_{t}[ \beta E_{t+1}[\pi_{t+2}] + \kappa \tilde{y}_{t+1}] + \kappa \tilde{y}_{t} \\
		&\ldots&	\\
		&=& \kappa \sum\limits_{j=0}^{\infty} \beta^{j} E_{t}[\tilde{y}_{t+j}]
\end{eqnarray*}

\end{frame}

%-------------------------------------------------------

%-------------------------------------------------------
	
\begin{frame}{Optimal Allocation}

From an alternative perspective\ldots
\begin{itemize}
\item	Assume $\pi_{t}=0$ $\forall t$ and all firms are initially at their desired markup
\item	If policy is such that marginal cost is stabilized, then the existing price will continue to be optimal
	\begin{itemize}
	\item	Since the markup is already at desired level and the marginal cost to which the markup is applied is unchanged
	\end{itemize}
\item	No firm (even the $1-\theta$ who \emph{can} reset) will want to change their price
	\begin{itemize}
	\item	Thus inflation will be zero
	\item	Price stickiness irrelevant (like being in flex price)
	\end{itemize}
\item	Output is equal to natural $\Rightarrow$ constant real marginal cost
	\begin{itemize}
	\item	Thus, under zero inflation, we have constant nominal marginal cost
	\item	Justifies firms not changing prices
	\end{itemize}
\end{itemize}

\vspace{2mm}
If policy achieves price stability then it also coincidentally achieves $y_{t}=y^{n}_{t}$
\begin{itemize}
\item	\textcolor{red}{`Divine coincidence'}
\item	No trade-off between price stability and goals for $y_{t}$
\end{itemize}

\end{frame}

%-------------------------------------------------------

%-------------------------------------------------------
	
\begin{frame}{Optimal Allocation}

Note that efficiency does not imply constant activity
\begin{itemize}
\item	$MC_{t}$ is stabilized such that the desired markup $\Rightarrow$ constant $P_{t}$
\item	But \emph{output} can still vary in an efficient allocation
	\begin{itemize}
	\item	$\tilde{y}_{t}=0 \Rightarrow y_{t} = y^{n}_{t}$
	\item	$y^{n}_{t}$ depends on $a_{t}$
	\end{itemize}
\item	This reflects one of the main insights of the RBC literature (business cycles $\centernot\implies$ market failure)
\end{itemize}

%Same markup and same p(t) means mc must be constant. If mpn(t) changes with yn(t) then w must be changing. If mpn(t) is constant then wage need not change in equilibrium.

\end{frame}

%-------------------------------------------------------

%-------------------------------------------------------
	
\begin{frame}{Optimal Allocation}

Nice quote from Gal\'i p. 104

\begin{quotation}
The intuition behind the desirability of zero inflation in the case of an efficient natural allocation can be conveyed as follows: if price stability is attained, then it must be the case that no firm is adjusting its price even when having the option to do so, from which it follows that the constraints on price setting are not binding and, hence, that the equilibrium allocation corresponds to that of an economy with flexible prices (which is, under the assumptions made here, efficient).
\end{quotation}

\end{frame}

%-------------------------------------------------------

%-------------------------------------------------------
	
\begin{frame}{Optimal Policy}

What interest rate policy is consistent with the optimal allocation as an \emph{equilbrium} outcome?
\begin{itemize}
\item	$y_{t}=y^{n}_{t}$ combined with the DIS curve implies $r_{t} = r^{n}_{t}$
\item	Zero inflation $\forall t$ implies $i_{t} =  r_{t}$ (by the Fisher equation)
\end{itemize}

\vspace{2mm}
Thus under optimal policy, in equilibrium,
\begin{equation}
i_{t} = r^{n}_{t}	\label{eqn:ir_rnt}
\end{equation}

\vspace{2mm}
But is this an adequate \emph{rule} for how the interest rate should be set in all contingencies?

\end{frame}

%-------------------------------------------------------

%-------------------------------------------------------
	
\begin{frame}{Optimal Policy}

$i_{t}=r^{n}_{t}$ holds in our desired equilibrium with $\tilde{y}_{t}=\pi_{t}=0$
\begin{itemize}
\item	But it \emph{also} can hold in other less desirable equilibria
\item	In these equilibria we do not have $\tilde{y}_{t}=\pi_{t}=0$
\item	Thus we lose the desired efficiency properties
	\begin{itemize}
	\item	We can have $i_{t}=r^{n}_{t}$ but if $\pi_{t}\neq0$ $\forall t$ then $r_{t}$ will deviate from $r^{n}_{t}$
	\item	Then we cannot guarantee that $y_{t}=y^{n}_{t}$ $\forall t$
	\end{itemize}
\end{itemize}

\vspace{2mm}
Equation (\ref{eqn:ir_rnt}) derived `assuming' optimal allocation ($\tilde{y}_{t}=\pi_{t}=0$ $\forall t$)
\begin{itemize}
\item	Doesn't allow for possibility of deviations from the optimal allocation
\item	This `opens the door' to alternative allocations
\item	Needs to be augmented with response to `off equilibrium' outcomes
\end{itemize}

\end{frame}

%-------------------------------------------------------

%-------------------------------------------------------
	
\begin{frame}{Optimal Policy}

Consider instead two alternative rules
\vspace{2mm}
\begin{itemize}
\item	A rule that responds to realized inflation and activity
\[
i_{t} = r^{n}_{t} + \phi_{\pi} \pi_{t} + \phi_{y} \tilde{y}_{t}	
\]
\item	A rule that responds to forecasts/expectations of inflation and activity
\[
i_{t} = r^{n}_{t} + \phi_{\pi} E_{t}[ \pi_{t+1} ] + \phi_{y} E_{t}[ \tilde{y}_{t} ]
\]
\end{itemize}

Let us simplify these rules further
\begin{eqnarray*}
i_{t} &=& r^{n}_{t} + \phi_{\pi} \pi_{t} 				\label{eqn:taylor_1} \\
i_{t} &=& r^{n}_{t} + \phi_{\pi} E_{t}[ \pi_{t+1} ]		\label{eqn:taylor_2}
\end{eqnarray*}

\end{frame}

%-------------------------------------------------------

%-------------------------------------------------------
	
\begin{frame}{Optimal Policy}

Explicit adjustments to the simple ($i_{t} = r^{n}_{t}$) policy if $\pi_{t}$ not as desired
\begin{itemize}
\item	$i_{t}=r^{n}_{t}$ \textbf{if} $\pi_{t}=0$ or $E_{t}[\pi_{t+1}]=0$, respectively, but\ldots
	\begin{itemize}
	\item	$\pi_{t}>0$ or $E_{t}[\pi_{t+1}]>0 \implies i_{t}>r^{n}_{t}$
	\item	$\pi_{t}<0$ or $E_{t}[\pi_{t+1}]<0 \implies i_{t}<r^{n}_{t}$
	\end{itemize}
\end{itemize}

\vspace{2mm}
Assume that $\phi_{\pi}>1$ (and, for the forecast rule, that $\phi_{\pi}$ is not `too big') 
\begin{itemize}
\item	In this case the \textbf{only} equilibrium possible is the desired one
\item	$\tilde{y}_{t}=\pi_{t}=0$ $\forall t$ so in equilibrium the adjustments never get made and $i_{t}=r^{n}_{t}$ after all!
\item	But the `threat' of those adjustments eliminates other equilibria
\end{itemize}

\vspace{2mm}
A plan for rates should specify actions \textbf{even in contingencies that should not occur} under the plan!

\end{frame}

%-------------------------------------------------------

%-------------------------------------------------------
	
\begin{frame}{Optimal Policy}

Some intuition for the importance of $\phi_{\pi}>1$ can be gained from the simplified forecast rule (where we ignore the response to the output gap)
\[
i_{t} = r^{n}_{t} + \phi_{\pi} E_{t}[ \pi_{t+1} ]
\]

\vspace{3mm}
Using the Fisher equation this implies that
\[
r_{t} = r^{n}_{t} + (\phi_{\pi}-1) E_{t}[ \pi_{t+1} ]
\]

\vspace{3mm}
$\phi_{\pi} >$ or $<1$ determines whether $r_{t}$ rises or falls with $E_{t}[\pi_{t+1}]$
\begin{itemize}
\item	Consider an example of an `inflation scare' to illustrate implications of this\ldots
\end{itemize}

\end{frame}

%-------------------------------------------------------

%-------------------------------------------------------
	
\begin{frame}{Optimal Policy}

If $\phi_{\pi}>1$, an increase in expected inflation $\implies r_{t}\uparrow$, all else equal
\begin{itemize}
\item	But we know $r_{t}\uparrow$ is contractionary and drives inflation down, so $\pi_{t+1}\downarrow$ in expectation
\item	But that \textbf{contradicts} assumption of higher inflation expectations!
\item	$\implies$ only $\pi_{t}=0$ $\forall t$ is consistent with equilibrium
\end{itemize}

\vspace{2mm}
If $\phi_{\pi}<1$, an increase in expected inflation $\implies r_{t}\downarrow$, all else equal
\begin{itemize}
\item	But we know $r_{t}\downarrow$ is expansionary and drives inflation up, so $\pi_{t+1}\uparrow$ in expectation
\item	That is \textbf{consistent with} assumption of higher inflation expectations
\item	$\implies \pi_{t}\neq 0$ is consistent with equilibrium
\end{itemize}

\end{frame}

%-------------------------------------------------------

%-------------------------------------------------------
	
\begin{frame}{Optimal Policy}


The `Taylor principle' ($\phi_{\pi}>1$) is desirable partly because it ensures policy responds `sufficiently strongly' to inflationary pressure
\begin{itemize}
\item	Note that $\phi_{\pi}>1$ ensures zero inflation in equilibrium
\item	But then we recover $i_{t}=r^{n}_{t}$ and the expectation response term is `dormant' in equilibrium
\item	Nevertheless, its presence is vital to eliminate other equilibria
\end{itemize}

\end{frame}

%-------------------------------------------------------

%-------------------------------------------------------
	
\begin{frame}{Simple Policy Rules}

One problem with specifying policy simply as $i_{t}=r^{n}_{t}$ is that it allows `multiple equilibria'
\begin{itemize}
\item	We saw a way to `fix' this was to specify `off equilibrium path' behavior
\end{itemize}
\vspace{2mm}	
But all of these approaches require knowledge of $r^{n}_{t}$
\begin{itemize}
\item	In practice, that's not easy (in fact, it's effectively impossible)
\item	See recent debates about `$r^{\ast}$' (pronounced $r$-star)
\end{itemize}
\vspace{2mm}	
Knowing $r^{n}_{t}$ requires exact knowledge of
\begin{itemize}
\item	The exact structure of the economy's `true model'
\item	The values taken by all its parameters (likely changing over time)
\item	The realized value of all the shocks that influence $r^{n}_{t}$
\end{itemize}

\end{frame}

%-------------------------------------------------------

%-------------------------------------------------------
	
\begin{frame}{Simple Policy Rules}

The previous rules are too `complicated' - hence people have proposed the use of `simple' rules
\vspace{2mm}
\begin{itemize}
\item	Informed by some of the same logic but\ldots
	\begin{itemize}
	\item	Depend only on observable variables
	\item	Don't require deep knowledge of (all the) structural parameters and shocks
	\end{itemize}
\vspace{2mm}	
\item	Will not be `optimal' but should perform `reasonably well'
	\begin{itemize}
	\item	The rules considered earlier were optimal but infeasible
	\end{itemize}
\vspace{2mm}
\item	Should be robust to a range of parameter values and sources of shocks
	\begin{itemize}
	\item	If being slightly wrong about a parameter is disastrous - then this is a bad rule!	
	\end{itemize}
\end{itemize}

\end{frame}

%-------------------------------------------------------

%-------------------------------------------------------
	
\begin{frame}{Simple Policy Rules}

We consider a simple `Taylor rule' (inspired by Taylor (1993))
\begin{equation}
i_{t} = \rho + \phi_{\pi} \pi_{t} + \phi_{y} \hat{y}_{t}
\end{equation}
where $\hat{y}_{t} \equiv y_{t} - y$ (log deviation from steady state - \textbf{not natural}) and $\phi_{\pi}$ and $\phi_{y}$ are set to ensure a unique equilibrium

\vspace{3mm}
Requires relatively little knowledge about the structure of the economy
\begin{itemize}
\item	Still assumes approximate knowledge of $\beta$ ($\rho$) and $\bar{y}$
\item	But see Levin \emph{et al} (1998) and Orphanides and Williams (2002, 2006) for `difference rules' that address this issue
\item	A related and very readable discussion of the role of rules is this \href{https://www.frbsf.org/economic-research/publications/economic-letter/2016/february/rules-of-engagement-monetary-policy-rules-speech/}{speech by Williams (2016))}
\end{itemize}

\end{frame}

%-------------------------------------------------------

%-------------------------------------------------------
	
\begin{frame}{Simple Policy Rules}

To assess the performance of the rule for a given parameterization we use
\[
\mathcal{L} \propto \left( \sigma + \frac{\varphi + \alpha}{1-\alpha} \right) var(\tilde{y_{t}}) + \frac{\varepsilon}{\lambda} var(\pi_{t})
\]

Welfare loss arises from $\pi_{t}\neq0$ and $y_{t}\neq y^{n}_{t}$
\begin{itemize}
\item	Derived via approximation to the welfare of representative household (Rotemberg and Woodford (1999))
\item	Weights are functions of the deep parameters
\item	Loss will be $>0$ (unless the rule replicates optimal policy)
\end{itemize}

%\vspace{2mm}
%An alternative would be to set the weights in an \emph{ad hoc} way
%\begin{itemize}
%\item	Central banks appear to dislike inflation and activity volatility
%\item	Weights not precisely determined by deep parameters
%\end{itemize}

\end{frame}

%-------------------------------------------------------

%-------------------------------------------------------
	
%\begin{frame}{Simple Policy Rules}
%
%Suppose technology shocks ($a_{t}$) are the only shocks hitting the economy
%\begin{itemize}
%\item	Tradeoff: stabilizing $y_{t}$ vs. stabilizing $\pi_{t}$ and the (welfare-relevant) $\tilde{y}_{t}$
%\item	$\phi_{y} \Uparrow \implies$Less volatile $y_{t}$ but more volatile $\pi_{t}$ and $\tilde{y}_{t}$
%\item	Welfare declines as $\phi_{y} \Uparrow$
%\item	Losses reduced if only respond to $\pi_{t}$ ($\phi_{y}=0$) and decline as $\phi_{\pi} \uparrow$
%\end{itemize}
%
%\vspace{2mm}
%Loose intuition for this tension\ldots
%\begin{itemize}
%\item	Recall that technology shocks tend to move output and inflation in \emph{opposite} directions\ldots
%\item	\ldots but the output \emph{gap} and inflation in the \emph{same} direction
%\item	Policy should loosen after a positive shock - but a (big enough) positive $\phi_{y}$ means policy tightens
%\end{itemize}
%
%\end{frame}
%
%%-------------------------------------------------------
%
%%-------------------------------------------------------
%	
%\begin{frame}{Simple Policy Rules}
%
%Suppose demand shocks ($z_{t}$) are the only shocks hitting the economy
%\begin{itemize}
%\item	No tradeoff: stabilizing $y_{t} \Leftrightarrow$ stabilizing $\pi_{t}$ and $\tilde{y}_{t}$
%\item	$\phi_{y} \Uparrow \implies$Less volatile $y_{t}$ \textbf{and} less volatile $\pi_{t}$ and $\tilde{y}_{t}$
%\item	Welfare improves as $\phi_{y} \Uparrow$ or as $\phi_{\pi} \Uparrow$
%\end{itemize}
%
%\vspace{2mm}
%Why the absence of a tradeoff?
%\begin{itemize}
%\item	$y^{n}_{t}$ is unaffected by demand shock ($z_{t}$) so output gap moves 1:1 with output
%\end{itemize}
%
%\vspace{2mm}
%These results suggest a sensible rule would be to respond fairly aggressively to inflation
%\begin{itemize}
%\item	Inflation response superior if there are supply shocks
%\item	Indifferent if there are no supply shocks
%\end{itemize}
%
%\end{frame}

%-------------------------------------------------------

%-------------------------------------------------------
	
\begin{frame}{Caveat}

Why not simply set $\phi_{\pi} \to \infty$?
\begin{itemize}
\item	Sometimes called the `inflation nutter' approach to policy
\item	In this simple model it essentially implements optimal policy
\end{itemize}

\vspace{2mm}
Beyond the scope of this course (see Ch. 5 if interested) but\ldots
\begin{itemize}
\item	We have been assuming that $y^{n}_{t}$ is efficient
\item	Means that price stability is consistent with ideal `activity' outcomes
\item	In richer models it may not be 
\end{itemize}

\vspace{2mm}
There may be reasons to weight price stability against variation in some measure of output or employment (or financial imbalances?)
\begin{itemize}
\item	Our model seems to be missing something
\item	No central bank thinks that focusing \textbf{purely} on price stability will achieve a desirable outcome
\end{itemize}

\end{frame}

%-------------------------------------------------------

%-------------------------------------------------------

\begin{frame}{Summary of conventional monetary policy}

\begin{itemize}
\item	Under the assumption of a subsidy that makes our economy's \emph{steady state} efficient, remaining inefficiencies arise from price stickiness
\item	This implies a role for policy to set rates such that price stability is ensured - resulting in output being equal to the natural rate in every period (which may vary over time with technology shocks)
\item	Under optimal policy (featuring zero inflation and output equal to natural) the nominal interest rate equals the natural real rate
\item	To ensure our desired equilibrium, policy should also specify how it will respond appropriately to deviations from desired outcomes
\item	But such policies often require an implausible degree of knowledge of the economy
\item	`Simple rules' may come close to achieving the same equilibrium but are implementable in the real world
\end{itemize}

\end{frame}