\documentclass[authoryear,11pt]{elsarticle}

%This eliminates the `Preprint submitted to...' footer on the first page
\makeatletter
\def\ps@pprintTitle{%
 \let\@oddhead\@empty
 \let\@evenhead\@empty
 \def\@oddfoot{}%
 \let\@evenfoot\@oddfoot}
\makeatother

\usepackage{amssymb}
\usepackage{amsthm}
\usepackage{caption}
\usepackage{amsmath}
\usepackage{morefloats}
\usepackage{bbm}        %To allow \mathbb{1}

\usepackage{rotating}   %To turn tables sidewaystable
\usepackage{graphicx}
\usepackage{setspace}
\usepackage{hyperref}

%\onehalfspacing

%\setlength{\parindent}{0pt}

\usepackage[top=3.5cm,bottom=3.75cm,left=2.45cm,right=2.45cm]{geometry}% by courtesy of Mico

\begin{document}
\begin{frontmatter}
\title{MFE Economics\\Lecture 2: Intuition}
\end{frontmatter}

%%%%%%%%%%%%%%%%%%%%%%%%%%%%%%%%%%%%%%%%%%%%%%%%%%%%%%%%%%%%%%%%%%%%%%%%%%%%%%%%%%%%%%%%%%%%%%%%%%%%%%%%%%%%%%%%%%%%%%%%%%%%%%%%%%%%%%%%%%%%%%%%%%%%%%%%
%%%%%%%%%%%%%%%%%%%%%%%%%%%%%%%%%%%%%%%%%%%%%%%%%%%%%%%%%%%%%%%%%%%%%%%%%%%%%%%%%%%%%%%%%%%%%%%%%%%%%%%%%%%%%%%%%%%%%%%%%%%%%%%%%%%%%%%%%%%%%%%%%%%%%%%%

\section{GE in an Endowment Economy}
This is the first part of a sequence of slides/lectures that build up towards the most complicated GE model we'll be covering, the New Keynesian model. We begin with the simplest possible - but still interesting and non-trivial - environment of an `endowment' economy (sometimes known as a Lucas Tree economy). There is no production, no aggregate saving technology, no labor supply - simply, people receive shares of the aggregate endowment (we initially index people with $i$) - imagine they have claims to fruit from a tree and the fruit will spoil overnight. Ultimately, we will show that (given our assumptions) the conditions for equilibrium will yield the same implications for equilibrium prices as in the case of an economy with a representative agent who has as her endowment the aggregate (or average - scale doesn't matter) endowment over $i$ and who behaves like a price-taking agent.\footnote{Price taking behavior is a bit more natural if we think of the representative is being the `average' person rather than the `aggregate' but the math doesn't change.} But initially we make explicit the presence of different agents with possibly different $y_{i}$ that are shares (that might vary by $i$) in the big aggregate fruit (endowment) tree.

The main point of the lecture is to introduce the idea of combining household optimality with the aggregate resource and savings constraints of the endowment economy (there's a tree, nothing influences its payoff and you can't save) to find prices that ensure equilibrium where the conditions for optimality and the constraints are all satisfied. We then call these equilibrium prices. On the side of constraints - you should be comfortable with the sequence of flow budget constraints (connecting savings brought into the period by an \textit{individual} household, consumption and savings leaving the period) and also with the intertemporal budget constraint that you should have derived in homework - along with the TVC and no-ponzi conditions. The derivation and interpretation/intuition of these is thoroughly covered in the lectures and homework answers. Ultimately, you have to satisfy budget constraints and we impose as a no-Ponzi condition that there is a limit to how quickly your consumption plan implies debt can grow over the infinite future. The limit means that the value of your future debt in the limit as the horizon $T\rightarrow \infty$, discounted appropriately back to the present, is weakly positive. People aren't going to give you a free lunch and allow you to take resources off them (in PV terms) indefinitely - you've got to pay the piper. You also will not choose to be on the other side - you will make sure your PV of wealth in the infinite future is not positive as if it were, you could consume more in all periods into the infinite future (if it were negative you are planning to oversave and are within your possibility set - so you can raise lifetime utility by eating more).

Each household receives an endowment that can be thought of as a realization of a share in the aggregate endowment. This represents the household's income in each period. They choose a plan for consumption (implying a plan for savings - given that their income is exogenous) which leads to the intertempral optimality condition between consumption in $t$ and $t+1$ given the market terms for transferring consumption between the two periods (or any two periods, really) i.e. given the interest rate. We solve for how households would respond to a given price of consumption today versus tomorrow (note in this bit we assume there is no risk) in terms of varying the `slope' of their consumption profile i.e. consumption tomorrow versus consumption today. A higher rate (better terms for saving) will be associated with a higher growth rate, under household optimality. There are various ways of deriving the Euler equation and you should be comfortable with all the methods discussed in the notes except the value function / Bellman equation approach on slide 13. That's way beyond the scope - but good to be aware of (maybe google \textit{Bellman's Principle of Optimality} if you have nothing better to do). Note that the Euler equation on its own doesn't fully nail down the household's optimal consumption for a given interest rate profile - you still need to know the levels as well as the slope and that, implicitly, entails being on the budget constraint boundary. Using a rule called `Walras' Law' however, we don't have to explicitly deal with that if we instead find prices ($R$) that ensures that aggregate consumption equals the aggregate endowment, which is a form of market clearing and technical feasibility assumption required for equilbrium.\footnote{Always remember, equilibrium requires optimality and feasibility, at the very least.}

There is no aggregate saving technology (like capital accumulation, apple storage or borrowing from abroad). As such, we know that in equilibrium aggregate consumption must equal aggregate endowment by feasibility and given that there can't be waste in the equilibrium of this model. But households don't care about aggregate feasibilty when they see a given price, $R$, they'll just solve their problems and to hell with the consequences! Thus, it is our desire to seek an equilbrium that then means that prices must take a particular value - the value that, when households are optimizing in response to it, ensures that $c_{t}=y_{t}$ (and by Walras' law - all budget constraints are binding). In practice - like when you are figuring out the value of $R$, this comes down to plugging $c_{t}=y_{t}$ into the Euler equation. The process for the aggregate endowment is a primitive of the model (if I tell you how the $y_{t}^{i}$ evolve and the form of the felicity function, you should be able to back out the required interest rate. You should also understand that the same equations can be satisfied in a world where there is a `representative' agent whose endowment is equal to the aggregate or average endowment of our multiple agent model. In this case the equilibrium prices will be the same, as well the \textit{aggregate} behavior of the economy.

\section{Representative Agent}
If you're confused by the concept of a representative agent and whether it means average (as the word `representative' suggests) vs. aggregate, imagine that we have uncountably many infinitely `small agents'. By small I mean these agents can't individually affect aggregates or prices in any - they are price takers. Thus it, may be useful to think of $i \in [0,K]$ with $K$ a finite real number strictly greater than zero. 

If $c(i)$ is the consumption of person $i$ then total consumption comes from `aggregating' or `adding up' (using the integral notation though in the class we used $\sum\limits_{i}$ in the lectures) $c = \int\limits_{0}^{K} c(i) di$. That's the aggregate, so what is the average? It's just
\[
c = \frac{\int\limits_{0}^{K} c(i) di}{K}
\]
so why not pick $K=1$? There are still uncountably infinitely many agents with $K=1$ (wikipedia properties of the real numbers) and the math is easier. It has the (unimportant) quirk that the average is `equal' to the aggregate but that's a trivial matter and dividing or not dividing by $K$ is just a normalization. Look at the expression (Euler equation) at the bottom of slide 23 - the scaling ($K$ or $1$) is just a normalization and cancels out in the Euler equation anyway.\footnote{For the mathematically inclined among you, note that this approach also gives a mathematically formal sense of `small' - all the households $i$ have zero measure.}

The point is that we can imagine a single agent who consumes as the aggregate or average and we can obtain the same expressions for prices and aggregate quantities.\footnote{On a deep level, this result depends on certain assumptions like `complete markets' allowing the sharing of idiosyncratic risks and thus eliminating them, but that's beyond the scope of this course.} Note that this doesn't mean there isn't some heterogeneity among the $i$ that might be interesting to some people - they may trade (among themselves) promises to share endowments etc. within a period, in return for reciprocation in other periods and some people may have larger shares of the aggregate endowment to begin with (lucky them!) in which case they will have higher consumption (relevant if you care about inequality). The point is that this heterogeneity does not affect prices and \textit{aggregate} properties of the economy, which is what we as macroeconomists care about.

Note that a very active area of research (google hank vs rank models) involves questioning whether disaggregated heterogeneity actually does matter for aggregate movements - i.e. whether the representative agent assumption is violated in ways that are relevant for macroeconomists. Does a monetary shock's effect depend on the distribution of income and wealth across agents in the economy? This and other questions are not yet answered but one we won't touch on them in our course.

\section{Ramsey Growth Model}
We now make things a bit more complicated than in the endowment economy by allowing for a (trivial - since it's assumed to be fixed/perfectly inelastic) labor supply decision on the part of the household and the presence of capital accumulation (though with $100\%$ depreciation) plus labor and capital hiring by firms with a Cobb-Douglas production function with a technology trend term $\theta_{t}$ that grows over time and is `labor augmenting' (pre-multiplies $L_{t}$ in the production function). Also, we just immediately assume a representative household - or, rather, we assume that the conditions for the existence of a representative household (complete markets is important here).

Perfectly inelastic supply means that we know from the get go that some constant labor (we normalize to $1$) is going to be part of the equilibrium outcome. We will want $L^{S}=L^{D}$ and we know $L^{S}$ is $1$. So it's a bit like the endowment economy intuition - there we needed to find prices ($R$) to ensure zero savings in the aggregate and here we'll need to find a wage (well, all prices matter, really) that ensures $L^{D}=1$. We will also need to find a price for the rental rate of capital, $r_{t}$ (firms rent capital from households). The $r_{t}$ also defines the `return' on capital (from $t-1$ investment) too, for the household, since all the capital in $t+1$ comes from investment in $t$ because with $100\%$ depreciation, none is left over from the capital operating in $t$.\footnote{So net investment = gross investment in this economy. More usually we have $K_{t+1} = (1-\delta)K_{t} + I_{t}$ where $\delta \approx 0.025$ i.e. a $2.5\%$ depreciation rate, so capital next period not only comes from investment today but there's a big chunk of slightly scratched/beaten up capital left over after depreciation.} This is why you see $r_{t+1}$ pop up in the Euler equation for the intertemporal optimality of the consumption/savings plan. Note that $r_{t}$ \textit{in equilibrium} will be equal to the marginal product of capital (from the firm's FOC) while the real wage will be equal to the marginal product of labor (from the firm's FOC). Note that the firm is a price taker in both these markets - as is the household.

Ultimately we can begin to combine all the equations (listed on slide 34 - where the first two bullets refer to the household and firm FOCs). In the example in class we see that the consumption output ratio is constant in any stable solution - if it's initially not that value, it will spiral off to +/- infinity (try putting a different value in and then use the middle equation on slide 35 to see what happens for future time periods\ldots

This is a growing economy in equilibrium so, while certain ratios (see slide 37) are constant, both the numerator and denominator are growing - necessarily at the same rate - the rate determined by the technology process. This model is thus designed to match various `stylized facts' about growth (most people would accept that in the long run, growth must come from technological progress - though how that comes about is unclear) and also some of the `great ratios' (see the Kaldor facts in the notes), such as fairly constant return to capital and labor share. Youshould be aware though that in recent decades there seems to have been a decline in GDP growth, technology growth and the labor share in many - especially advanced - economies. This is related to the secular stagnation debate and the labor share decline is a very poorly understood puzzle and an area of active research. Be aware of the general stylized facts I refer to in the slides - they are things you should broadly know - you don't need to know every last decimal or what a particular country's (except maybe US or your home country!) is.

\section{Monetary Models - Background}
This is already fairly clear in the slides and you should certainly read the first chapters of Gali and Walsh's textbooks. The introductions to the Romer and Romer papers and the Nakamura Steinnson papers should also be pretty readable. Be aware of the classical dichotomy - what it means and how plausible / what evidence is cited for it holding in the short and long run. As economists you should understand the general progress (?) made from very loose arguments in favor of the effect (and, if it has any, its desirability) of monetary policy on the economy, how the RBC literature revolutionized macro and introduced careful micro-founded DSGE approaches - which were then picked up by NK practitioners and integrated into models with non-trivial effects of monetary policy on real variables and a justification for why monetary policy \textit{should} try to influence the economy.


\end{document}

