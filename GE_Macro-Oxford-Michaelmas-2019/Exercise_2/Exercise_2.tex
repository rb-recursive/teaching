\documentclass[authoryear,11pt]{elsarticle}

%This eliminates the `Preprint submitted to...' footer on the first page
\makeatletter
\def\ps@pprintTitle{%
 \let\@oddhead\@empty
 \let\@evenhead\@empty
 \def\@oddfoot{}%
 \let\@evenfoot\@oddfoot}
\makeatother

\usepackage{amssymb}
\usepackage{amsthm}
\usepackage{caption}
\usepackage{amsmath}
\usepackage{morefloats}
\usepackage{bbm}        %To allow \mathbb{1}

\usepackage{rotating}   %To turn tables sidewaystable
\usepackage{graphicx}
\usepackage{setspace}
\usepackage{hyperref}

%\onehalfspacing

\setlength{\parindent}{0pt}

\usepackage[top=3.5cm,bottom=3.75cm,left=2.45cm,right=2.45cm]{geometry}% by courtesy of Mico

\begin{document}
\begin{frontmatter}
\title{MFE Economics\\Problem set 2}
\end{frontmatter}

%%%%%%%%%%%%%%%%%%%%%%%%%%%%%%%%%%%%%%%%%%%%%%%%%%%%%%%%%%%%%%%%%%%%%%%%%%%%%%%%%%%%%%%%%%%%%%%%%%%%%%%%%%%%%%%%%%%%%%%%%%%%%%%%%%%%%%%%%%%%%%%%%%%%%%%%
%%%%%%%%%%%%%%%%%%%%%%%%%%%%%%%%%%%%%%%%%%%%%%%%%%%%%%%%%%%%%%%%%%%%%%%%%%%%%%%%%%%%%%%%%%%%%%%%%%%%%%%%%%%%%%%%%%%%%%%%%%%%%%%%%%%%%%%%%%%%%%%%%%%%%%%%

\section{Deriving the PV budget constraint}
In class we had the flow budget constraint of the form below\footnote{As someone pointed out in the lecture, it's sort of overcomplicating things by carrying around $t$ to mean a generic period - why not simply assume $t=0$ and get on with it. It's basically because we want to emphasize that these flow budget constraints do apply generically, in all periods.}
\begin{equation}
a_{t+s+1} = R_{t+s} a_{t+s} + y^{i}_{t+s} - c^{i}_{t+s} \; \forall s \geq 0 \label{eqn:flow}
\end{equation}
\begin{enumerate}
\item	In period $t+s-1$ you previously decided what your wealth, $a_{t+s}$, should be in your `bank account' at the end of that period.
\item	On entering period $t+s$ your wealth from the previous period earns the gross interest rate, $R_{t+s}$, meaning you have $R_{t+s} a_{t+s}$ before you receive your endowment $y_{t+s}$ and purchase consumption, $c_{t+s}$ in period $t+s$.
\item	Your savings in period $t+s$ (or if negative, dissavings) are $y_{t+s} - c_{t+s}$ which, combined with the wealth $R_{t+s} a_{t+s}$ will determine the wealth at the end of $t+s$, $a_{t+s+1}$, which you carry forward into $t+s+1$.
\item	Go to the next period and we're back to the start of the loop, but one period in the future.
\end{enumerate}

\begin{itemize}
\item	Rearrange equation \ref{eqn:flow} to have $a_{t+s}$ on the LHS and everything else on the right
\item	Assume that we start in $t=0$ (let's just get rid of $t$ for now by assuming it's period `0')
\item	By repeatedly substituting for future values of $a$ (keep using the next period version of the equation you just derived, to eliminate  wealth from the RHS) show that\footnote{Hint: Do it for the first few periods, rearrange terms/tidy up and you'll see the pattern emerging\ldots Also, note that the $\tilde{R}$ thing is just a multi-period PV discount allowing for possibly varying $R_{s}$.}
\begin{eqnarray}
a_{0} + \sum\limits_{s=0}^{T-1} \frac{y_{s}}{\tilde{R}_{s}} &=& \frac{1}{\tilde{R}}a_{T} + \sum\limits_{s=0}^{T-1} \frac{c_{s}}{\tilde{R}_{s}} \label{eqn:pv_bc_T} \\
\tilde{R}_{K} &\equiv& \prod\limits_{s=0}^{K} R_{s} \label{eqn:pv_bc_T}
\end{eqnarray}
\item	In class, we discussed a TVC that means that, as $T\rightarrow\infty$, the $\frac{1}{\tilde{R}}a_{T}$ terms is taken to be zero and we obtain the infinite horizon \textit{present value budget constraint}, below:
\begin{equation}
a_{0} + \sum\limits_{s=0}^{\infty} \frac{y_{s}}{\tilde{R}_{s}} = \sum\limits_{s=0}^{\infty} \frac{c_{s}}{\tilde{R}_{s}} \label{eqn:pv_bc_infty}
\end{equation}
\item	For finite $T$ and assuming $a_{0}$ was positive, what does a negative $a_{T}$ imply about the general relationship between $y$ and $c$ in the intervening periods? What does it imply about the relationship in any single period (assuming $T>1$)?
\item	If $$a_{0} + \sum\limits_{s=0}^{T} \frac{y_{s}}{\tilde{R}_{s}} < \sum\limits_{s=0}^{T} \frac{c_{s}}{\tilde{R}_{s}}$$ then, using equation (\ref{eqn:pv_bc_infty}), show that implies equation (\ref{eqn:pay_piper}) below. Interpret what that means for the general relationship between the PV of $y$ and $c$ from $T+1$ onwards? Relate it to the phrase `pay the piper'.
\begin{equation}
\sum\limits_{s=T+1}^{\infty} \frac{y_{s}}{\tilde{R}_{s}} > \sum\limits_{s=T+1}^{\infty} \frac{c_{s}}{\tilde{R}_{s}} \label{eqn:pay_piper}
\end{equation}
\end{itemize}

\section{Optional question: CRRA utility}
The Arrow-Pratt measure of absolute risk aversion for an agent with preferences representable using a suitably differentiable felicity function, $u(\cdot)$ pre is given by
\[
A(c) = - \frac{u''(c)}{u'(c)}
\]
where $u'(\cdot)$ indicates the first derivative of $u$ and $u''(\cdot)$, the second.

The Arrow-Pratt measure of relative risk aversion for such an agent is given by
\[
R(c) = c \times A(c)
\]

Calculate these two measures of risk aversion for an agent with felicity function
\[
u(c) = \frac{c^{1-\sigma} - 1}{1-\sigma}
\]
This function is typically used whenever one discusses relative risk aversion and it is referred to as a `constant relative risk aversion' (CRRA) preference specification. Why does it have this name?

Also, calculate relative risk aversion for an agent with a similar felicity function
\[
u(c) = \frac{c^{1-\sigma}}{1-\sigma}
\]
Compare to your previous answer and comment (very) briefly.

\section{Optional question (for revision/background): CRRA and Log Utility}
L'Hopital's rule states (roughly) that, if\ldots
\begin{itemize}
\item	$\lim_{x \to c} f(x) = \lim_{x \to c} g(x)=0$
\item	$g'(x)\neq0$
\item	$\lim_{x \to c} \frac{f'(x)}{g'(x)}$ exists
\end{itemize}
then
\[
\lim_{x \to c} \frac{f(x)}{g(x)} = \lim_{x \to c} \frac{f'(x)}{g'(x)}
\]

Frequently, you will find papers that say something like, we consider an agent with CRRA preferences $u(c) = \frac{c^{1-\sigma} - 1}{1-\sigma}$ for $\sigma>1$ and $u(c) = \log{(c)}$ for the case of $\sigma=1$. It is not immediately obvious (to most people) why this should make sense as $\sigma$ does not seem to appear in any meaningful way. The reason is that $\log{(c)}$ is the limit of $\frac{c^{1-\sigma} - 1}{1-\sigma}$ as $\sigma \rightarrow 1$. Use L'Hopital's rule to show this.\footnote{Hint: rewrite $C_{t}$ as $e^{\log{(C_{t})}}$ in the CRRA utility function.} Referring to the previous question, why, therefore do we tend to have the $-1$ in the definition of the CRRA felicity function, despite your answer to the final bit of that question?

\section{GE model with CRRA felicity}
Review the arguments in the second lecture showing that the market interest rate in the GE model with heterogenous endowments (possibly different ${y_{t}^{i}}_{i=1:N}$ for different people $i$) is equal to the market interest rate in a GE model with a representative agent who receives an endowment process equal to $y_{t}=\sum\limits_{i=1}^{N}y_{t}^{i}$. The general equilibrium result in class (the endowment economy case) assumed that felicity was logarithmic in consumption. Will the market rate of interest be the same in GE models with and without heterogeneity if the felicity function is of the CRRA form $u(c_{t}^{i})=\frac{(c_{t}^{i})^{1-\sigma }}{1-\sigma }$?


\section{Optional question (for revision): Elasticities}
As a quick refresher, if we have a function $f(x)$ then its \href{https://en.wikipedia.org/wiki/Elasticity_of_a_function}{\textbf{elasticity}} with respect to $x$ is defined as
\[
 \frac{x}{f(x)} \frac{df(x)}{dx}
\]
which gives the percentage change in $f$ for a percentage change in $x$ (for small changes). In fact, one can alternatively calculate it as
\[
\frac{d \log{(f(x))}}{d \log{(x)} }
\]

Now, consider the production function of firm $i$
\[
Y_{i,t} = A_{t} N_{i,t}^{1-\alpha}
\]
What is the elasticity of output with respect to labor?

\section{CRRA and Elasticity of Intertemporal Substitution}
Using the Euler equation from a CRRA agent
\[
1 = \beta R \left( \frac{c^{i}_{t+1}}{c^{i}_{t}} \right)^{-\sigma}
\]
Show that the elasticity of $G_{c,t+1} \equiv \frac{c_{t+1}}{c_{t}}$ with respect to $R$ is $\frac{1}{\sigma}$. Note that we are in a riskless case here (hence the absence of an expectation operator in the Euler equation). Comment on the appearance and interpretation of $\sigma$ in the elasticity, given the absence of risk in this case.

\section{Pricing with market power}
\textbf{[RB: This question is to give you a head start on some NK stuff in a non-NK setting]}

In the NK model we are about to cover in class, we assume that firms exist in a situation of `monopolistic competition'. Without delving into the details of monopolistic competition (see various lecture notes available all over the web) what is important for our purposes is that the firms have some pricing power - they can vary their price marginally and thereby induce marginal changes in demand for their good. People sometimes refer to this as `facing a downward sloping demand curve'.\footnote{This is in contrast to the perfect competition case where firms take prices as given and if their price - somehow - ever deviated from that price, they would lose all demand or absorb the whole market, neither of which is sustainable in an equilibrium. Thus they effectively face a `horizontal' demand curve at whatever the competitive market price is.} If the firm wants to sell more, it must cut its price and if it wants to sell less, it must raise its price. Ultimately, then, their profit maximizing decision about the scale of operation (and implicitly employment etc.) comes down to choosing a price. They are price setters, not price takers.

In monopolistic competition, people normally have in mind a situation in which each monopolist produces a particular good that is somewhat differentiated but also somehow similar to goods produced by the other monopolists. The fact that the goods are different but somewhat substitutable means that raising the price may drive people towards the other goods. For example, a firm may have a monopoly in producing soft drink `x' but if they raise the price, that partly leads people to substitute towards soft drink `y' (a similar, if differentiated, good).\footnote{An important element of `competition' comes from firms entering or exiting the industry until expected profits from entry are zero - but we ignore this industrial organization stuff\ldots}

In the NK model, you will see that the particular structure of household preferences where the consumer gets utility from a bundle of different consumption goods gives rise to a degree of substitutability across goods, which is reflected in the demand curve for each of the different goods. These curves show that demand is decreasing in the goods' relative price where the steepness of that decrease is controlled by the parameter, $\varepsilon$, which captures the cross price elasticity of substitution (how willing a consumer is to substitute towards a cheaper good). In this homework example, we strip away all the NK background and consider a related but much simpler pricing problem for a single firm. This should prepare you for when this sort of thing is embedded in the richer NK model.

\vspace{5mm}
Let us assume that a firm faces a demand curve of the following form
\[
Y_{t} = P_{t}^{-\varepsilon}
\]
and that they have access to a production technology
\[
Y_{t} = A_{t}N_{t}^{1-\alpha}
\]

\begin{itemize}
\item	The production function implies a function $\mathcal{N}$ that gives the amount of labor required to produce a certain amount of output (assuming $A_{t}$ is given), i.e. $N_{t} = \mathcal{N}(Y_{t})$. Derive that function.
\item	Using the demand curve, show how demand changes for a marginal change in price (i.e. $\frac{dY_{t}}{dP_{t}}$).
\item	What is the price elasticity of demand (i.e. $-\frac{dY_{t}}{dP_{t}}\frac{P_{t}}{Y_{t}}$)?
\item	Noting that total revenue is quantity times price, how does it change with a marginal change in price?
\item	Noting that total cost is wage times hours, how does it change with a marginal change in price?
\item	At an optimum, the change in profits from a marginal change in price should be zero - use this and your previous results to show the markup of price over marginal cost (i.e. the change in costs from a marginal change in output) at the optimum.\footnote{Hint: It should be $\mathcal{M}\equiv \frac{\varepsilon}{\varepsilon-1}$}
\item	Briefly comment on how this differs conceptually from the price-taking perfectly competitive case?
\end{itemize}

%\section{Harder question: Correcting an inefficiency with a Pigouvian subsidy}
%Consider the monopolistically competitive firm discussed above. The chosen output level (reflecting the chosen price) is such that price is a markup on marginal cost. This is inefficient. Effectively, the price reflects the marginal valuation of the output by the marginal consumer. Since the price is above marginal cost it means that at $Y^{\ast}$ (the output level implied by the chosen price, $P^{\ast}$, given the demand curve) the `consumer' values an extra marginal unit more than the cost to the firm of producing that marginal unit (the marginal cost), thus there are possible gains from trade that are not being exploited - the consumer would be willing to pay a price between $MC(Y^{\ast})$ and $P(Y^{\ast})$ for that good. The firm would be fine with that and so would the consumer - implying a Pareto improvement. This will be possible right up to the point, call it $Y^{Eff}$, where we have price = marginal cost.
%
%For efficiency we would like to somehow induce the firm to produces $Y^{Eff}>Y^{\ast}$. Note that the firm produces to the point at which marginal cost = marginal revenue where the latter lies below the demand curve. One can offer a subsidy to the firm to change the `private' marginal cost it faces. In particular, if the `government' offers to pay fraction $\tau$ of the wage then for a given wage received by a worker, $W_{t}$, the firm only pays $(1-\tau)W_{t}$ - so from the firm's perspective this lowers their marginal cost. Call this private marginal cost $\widehat{MC}(Y_{t}) = \frac{(1-\tau)W_{t}}{MPN_{t}} = (1-\tau)MC(Y_{t})$ (recall marginal cost is wage paid by firm divided by the marginal product of labor, at the firm's privately optimal choice).
%
%What value should $\tau$ take to induce the firm to produce $Y^{Eff}$? HINT: The firm will still produce to the point where its \emph{private} MC = MR

\end{document}

